\documentclass[a4paper,12pt]{article}
\usepackage[utf8]{inputenc}
\usepackage{ngerman}

%opening
\title{The Vibrating Reed}
\author{Johannes Haux}

\begin{document}

\maketitle

\begin{abstract}
This lab course analyses the characteristics of a vibrating reed. Aim is to not only understand the physical principals leading to the obtained results, but also to get an understanding of the technical principals of data aquisition. Using LabView we comunicate with the various instruments, manipulating settings and reading out measurements. 
In a first part we find the resonance frequency $\omega_R$ of the vibrtating reed and its first two harmonics, in a second we determine the temperature dependecy of $\omega_R(T)$.
\end{abstract}

\section{Introduction}
\label{sec:intro}

\section{Theory}
\label{sec:theo}


\subsection{Driven and damped harmonic oszilator with the example of a vibrating reed}
\label{sec:osz}
By applying the following differtial equation we can describe a harmonic oszilator, driven by the force $F = F_0 cos(\omega t)$ and damped by the dampening factor $\gamma$ and with its Eigen-frequency $\omega_0$:
\begin{equation}
 \frac{\partial^2z}{\partial t^2} + \gamma \frac{\partial z}{\partial t} + \omega_0^2 = f_0 \mathrm{cos}(\omega t)
 \label{eq:stAmp}
\end{equation}
$\omega$ is the frequency and $f_0 = F_0/m$ is the Amplitude of the driving force, the latter normed to its mass. 

Some helpful equations can be derived from this:
By assuming that $z = \zeta e^{i(\omega t - \phi}$ we find the stationary solution for the Amplitude $\zeta (\gamma)$:
\begin{equation}
 \zeta = \frac{f_0}{\sqrt{(\omega_0^2 - \omega^2)^2 + \gamma^2\omega^2}}
\end{equation}
It has the shape of a Lorentzian function.
It is also possible to find the phasedifference of driving force and oszillator:
\begin{equation}
 \mathrm{tan}(\phi) = \frac{\gamma\omega}{\omega_0^2 - \omega^2}
\end{equation}

Equation \ref{eq:stAmp} reaches its maximum at
\begin{equation}
 \omega_R = \sqrt{\omega_0^2 - \frac{\gamma^2}{2}} \mathrm{,}
\end{equation}
the resonance frequency.

The attenuation of the system is described by the quality $\mathcal{Q}$ of the oszilator:
\begin{equation}
 \mathcal{Q} = 2\pi\frac{\omega_0}{\gamma}
\end{equation}


As described in \ref{bib:cla} the oszilation of a vibrating reed can be described as follows:
\begin{equation}
 \frac{\partial^2 z}{\partial t^2} = -\frac{d^2}{12} v_Y^2 \frac{\partial^4 z}{\partial x^4}
\end{equation}
Here $x$ is the coordinate of a small portion along the axis of the reed, $z$ is the transversal displacement, $t$ is the time, $d$ the thickness of the reed and $v_Y$ is the speed of sound inside the material the reed is made of. For a better understanding see figure \ref{fig:reed}.
As 
\begin{equation}
 v_Y = \sqrt{\frac{E}{\rho}}
\end{equation}
holds, we see that $v_Y$ depends on the mass density $\rho$ and the modulus of elasticity $E$.
This way we can determine the different Eigen-frequencies $\nu_n$
\begin{equation}
 \nu_n = \alpha_n (2n+1)^2 \frac{\pi}{16\sqrt{3}} \frac{d}{l^2} v_Y
\end{equation}

$\alpha_n$ is calculated numerically:
\begin{eqnarray}
 \alpha_0 = 1,424987 \mathrm{,} \alpha_1 = 0,992249  \mathrm{,} \alpha_2 = 1,000198 \mathrm{,} \alpha_3 = 0,999994 \mathrm{,}
\end{eqnarray}

 $\alpha_n \approx 1$ for all other $n$.



\subsection{}

\section{Setup}
\label{sec:set}

\subsection{Vibrating Reed}
\label{sec:reed}


\section{Methods}
\label{sec:meth}


\section{Results}
\label{sec:res}


\section{Discussion}
\label{sec:disk}


\end{document}

\bibliography{CuD_long}

