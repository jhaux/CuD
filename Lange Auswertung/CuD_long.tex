\documentclass[a4paper,12pt]{article}
\usepackage{multicol}
\usepackage{amsmath}
\usepackage[utf8]{inputenc}
\usepackage[english]{babel}

%opening
\title{The Vibrating Reed}
\author{Johannes Haux}

\begin{document}

\maketitle

\begin{abstract}
This lab course analyses the characteristics of a vibrating reed. Aim is to not only understand the physical 
principals leading to the obtained results, but also to get an understanding of the technical principals of 
data aquisition. Using LabView we comunicate with the various instruments, manipulating settings and reading out measurements. 
In a first part we find the resonance frequency $\omega_R$ of the vibrtating reed and its first two harmonics, in a 
second we determine the temperature dependecy of $\omega_R(T)$.
\end{abstract}
\vspace{2em}

%\begin{multicols}{2}

\section{Introduction}
\label{sec:intro}

\section{Theory}
\label{sec:theo}

To get a good understanding of the underlying processes that occur in the experiment we need to understand how a damped harmonic
oscillator under an external force behaves. We will derive how to calculate the Eigenvalues of the frequencies of an harmonic oscillator to later
estimate their actual values.

\subsection{Driven and damped harmonic oszilator with the example of a vibrating reed}
\label{sec:osz}
By applying the following differential equation we can describe a harmonic oscillator, driven by the force $F = F_0 cos(\omega t)$ and damped by the dampening factor $\gamma$ and with its Eigen-frequency $\omega_0$:
\begin{equation}
 \frac{\partial^2z}{\partial t^2} + \gamma \frac{\partial z}{\partial t} + \omega_0^2 = f_0 \mathrm{cos}(\omega t)
 \label{eq:stAmp}
\end{equation}
$\omega$ is the frequency and $f_0 = F_0/m$ is the Amplitude of the driving force, the latter normed to its mass. 

Some helpful equations can be derived from this:
By assuming that $z = \zeta e^{i(\omega t - \phi}$ we find the stationary solution for the Amplitude $\zeta (\gamma)$:
\begin{equation}
 \zeta = \frac{f_0}{\sqrt{(\omega_0^2 - \omega^2)^2 + \gamma^2\omega^2}}
\end{equation}
It has the shape of a Lorentzian function.
It is also possible to find the phase difference of driving force and oscillator:
\begin{equation}
 \mathrm{tan}(\phi) = \frac{\gamma\omega}{\omega_0^2 - \omega^2}
\end{equation}

Equation \ref{eq:stAmp} reaches its maximum at
\begin{equation}
 \omega_R = \sqrt{\omega_0^2 - \frac{\gamma^2}{2}} \mathrm{,}
\end{equation}
the resonance frequency.

The attenuation of the system is described by the quality $\mathcal{Q}$ of the oszilator:
\begin{equation}
 \mathcal{Q} = 2\pi\frac{\omega_0}{\gamma}
\end{equation}


As described in \ref{bib:cla} the oszilation of a vibrating reed can be described as follows:
\begin{equation}
 \frac{\partial^2 z}{\partial t^2} = -\frac{d^2}{12} v_Y^2 \frac{\partial^4 z}{\partial x^4}
\end{equation}
Here $x$ is the coordinate of a small portion along the axis of the reed, $z$ is the transversal displacement, $t$ is the time, $d$ the thickness of the reed and $v_Y$ is the speed of sound inside the material the reed is made of. For a better understanding see figure \ref{fig:reed}.
As 
\begin{equation}
 v_Y = \sqrt{\frac{E}{\rho}}
\end{equation}
holds, we see that $v_Y$ depends on the mass density $\rho$ and the modulus of elasticity $E$.
This way we can determine the different Eigen-frequencies $\nu_n$
\begin{equation}
 \nu_n = \alpha_n (2n+1)^2 \frac{\pi}{16\sqrt{3}} \frac{d}{l^2} v_Y
\end{equation}

$\alpha_n$ is calculated numerically:
\begin{align}
 \alpha_0 &= 1,424987 \mathrm{,}\; \alpha_1 = 0,992249 \mathrm{,} \\ 
 \alpha_2 &= 1,000198 \mathrm{,}\; \alpha_3 = 0,999994 \mathrm{,} \\
 \alpha_n &\approx 1 \text{ for all other } n.
\end{align}


Applying a force on the damped harmonic oscillator, that attacks only at the end of the reed ($x = l$), we can formulate the following differential equation:
\begin{equation}
 \frac{\partial^2z}{\partial t^2} + \frac{d^2}{12} v_Y \frac{\partial^4z}{\partial x^4} = f_0 e^{i\omega t}\delta(x - l)
\end{equation}

With only a small damping factor we can approximate the vibration modes around the resonance frequencies with a Lorentzian.
The relative amplitude can then be approximately described by 

\begin{align}
 \left| \frac{\zeta(\omega)}{\zeta(\omega_{R,n})} \right| &\approx \frac{\gamma \omega_{0,n)}}{\sqrt{(\omega_{0,n}^2 - \omega^2)^2 + \gamma^2\omega_{0,n}^2}} \\
 n &= 0, 1, 2,\dots \; \text{.}
\end{align}






\section{Setup}
\label{sec:set}
% Schöne Graphiken hier
% Ausführlich weil einfach!

{\LARGE Graphik}

The core element of our experiment is the vibrating reed.  It is a small metal plate that is fixed to the ground. The free end is placed between 
two electrodes as can be seen in figure \ref{fig:set}.



\subsection{}

\subsection{Parameter estimation}
% Abschätzungen der verschiedenen Parameter

Prior to the experiment we need to estimate certain parameters to the reed, we use, be abled to run the experiment. These are the maximum Voltage $U_{d,max}$, which shows, 
how small our expected signals are and the base Eigenfrequency $\nu_0$, where we have to look for the Lorentz-peaks, we want to observe.


% U_d,maximum
The maximum
% nu_0


\subsection{Vibrating Reed}
\label{sec:reed}
% ?? Sollte im Theorieteil abgearbeitet sein

\section{Methods}
\label{sec:meth}
% Hier kommen die genutzten VIs rein und die Fitmethoden


\section{Results}
\label{sec:res}
% Kurz und knackig
% Alle aussagekräftigen graphen

\section{Discussion}
\label{sec:disk}
% Wieder ausführlich
% Siehe Protokoll

%\end{multicols}
\end{document}

\bibliography{CuD_long}

