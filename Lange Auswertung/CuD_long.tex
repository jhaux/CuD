\documentclass[twoside, a4paper, DIV=11,twocolumn, 12pt]{book}
% \documentclass[oneside, a4paper, DIV=11]{book}

% PACKAGES
\usepackage[english]{babel}
\usepackage[utf8]{inputenc}
\usepackage[singlespacing]{setspace} % 1,5 Zeilenabstand: onehalfspacing
\usepackage[nottoc]{tocbibind} % Literaturverzeichnis im Inhaltsverzeichnis
%\usepackage{subscript} % Tiefegestellter Text außerhalb des Mathemodus
\usepackage{fixltx2e} % Tiefegestellter Text außerhalb des Mathemodus
\usepackage{amsmath,
	    amssymb,
	    array,
	    balance, % gleich lange säulen am ende jedes Kapitels
	    color,
	    graphicx,
	    hyphsubst, % Silbentrennung wie ich es will
	    hyperref,
	    natbib,
	    ragged2e,
	    sectsty,
	    siunitx, 
	    tabularx,
	    tikz}

% \usepackage{pdflscape,}
%\hypersetup{linktocpage} % nur die Seitenzahlen erscheinen als Link
\usepackage{./src/mymacros}
\usepackage{./src/CuDmacros}

\usepackage{etoolbox} % citations in italic:
\makeatletter
\patchcmd{\NAT@test}{\else\NAT@nm}{\else\NAT@nmfmt{\NAT@nm}}{}{}
\let\NAT@up\itshape
\makeatother


% \usepackage{showframe} % show margins and stuff

% SETTINGS

\hypersetup{
    colorlinks,
    citecolor=black,
    filecolor=black,
    linkcolor=black,
    urlcolor=black
}

\sisetup{
    locale=DE,
    loctolang={DE:ngerman},
    load-configurations = binary,
    binary-units = true,
    list-final-separator = { \translate{und} },
    list-units=single,
    range-phrase = { \translate{bis} },
    range-units=single,
    round-mode = places,
    round-precision = 3
}

\pagestyle{plain}

% sectsty setup
\chapterfont{\raggedright}
\sectionfont{\raggedright}

%TITLE
\title{The Vibrating Reed Experiment}
\author{Johannes Haux}
\date{1.4.2015}

\begin{document}
\onecolumn
\maketitle
\pagenumbering{roman}

\chapter*{\,}

 %Deutsch
  \noindent \textbf{Abstract} \hspace{0.3cm} 
This lab course analyses the characteristics of a vibrating reed. Aim is to not only understand the physical 
principals leading to the obtained results, but also to get an understanding of the technical principals of 
data acquisition. Using LabView we communicate with the various instruments, manipulating settings and reading out measurements. 
In a first part we find the resonance frequency $\omega_R$ of the vibrating reed and its first two harmonics, in a 
second we determine the temperature dependency of $\omega_R(T)$.

\emptypage

\tableofcontents

\emptypage

\listoffigures

\newpage

% jetzt gehts los
\pagenumbering{arabic}
\twocolumn
\balance % gleich lange säulen am ende jedes Kapitels
\chapter{Introduction}
\label{sec:intro}

\chapter{Theory}
\label{sec:theo}

To get a good understanding of the underlying processes that occur in the experiment we need to understand how a damped harmonic
oscillator under an external force behaves. We will derive how to calculate the Eigenvalues of the frequencies of an harmonic oscillator to later estimate their actual values.

This whole section is a short english version of the german equivalent in the script of this experiment: \cite{fp75}.

\section{Driven and damped harmonic oscillator with the example of a vibrating reed}
\label{sec:osz}
By applying the following differential equation we can describe a harmonic oscillator, driven by the force $F = F_0 cos(\omega t)$ and damped by the dampening factor $\gamma$ and with its Eigen-frequency $\omega_0$:
\begin{equation}
 \frac{\partial^2z}{\partial t^2} + \gamma \frac{\partial z}{\partial t} + \omega_0^2 = f_0 \mathrm{cos}(\omega t)
 \label{eq:stAmp}
\end{equation}
$\omega$ is the frequency and $f_0 = F_0/m$ is the Amplitude of the driving force, the latter normed to its mass. 

Some helpful equations can be derived from this:
By assuming that $z = \zeta e^{i(\omega t - \phi}$ we find the stationary solution for the Amplitude $\zeta (\gamma)$:
\begin{equation}
 \zeta = \frac{f_0}{\sqrt{(\omega_0^2 - \omega^2)^2 + \gamma^2\omega^2}}
\end{equation}
It has the shape of a Lorentzian function.
It is also possible to find the phase difference of driving force and oscillator:
\begin{equation}
 \mathrm{tan}(\phi) = \frac{\gamma\omega}{\omega_0^2 - \omega^2}
\end{equation}

Equation \ref{eq:stAmp} reaches its maximum at
\begin{equation}
 \omega_R = \sqrt{\omega_0^2 - \frac{\gamma^2}{2}} \mathrm{,}
\end{equation}
the resonance frequency.

The attenuation of the system is described by the quality $\mathcal{Q}$ of the oscillator:
\begin{equation}
 \mathcal{Q} = 2\pi\frac{\omega_0}{\gamma}
\end{equation}


As described in \ref{bib:cla} the oscillation of a vibrating reed can be described as follows:
\begin{equation}
 \frac{\partial^2 z}{\partial t^2} = -\frac{d^2}{12} v_Y^2 \frac{\partial^4 z}{\partial x^4}
\end{equation}
Here $x$ is the coordinate of a small portion along the axis of the reed, $z$ is the transversal displacement, $t$ is the time, $d$ the thickness of the reed and $v_Y$ is the speed of sound inside the material the reed is made of. For a better understanding see figure \ref{fig:reed}.
As 
\begin{equation}
 v_Y = \sqrt{\frac{E}{\rho}}
\end{equation}
holds, we see that $v_Y$ depends on the mass density $\rho$ and the modulus of elasticity $E$.
This way we can determine the different Eigenfrequencies $\nu_n$
\begin{equation}
 \nu_n = \alpha_n (2n+1)^2 \frac{\pi}{16\sqrt{3}} \frac{d}{l^2} v_Y
\end{equation}

$\alpha_n$ is calculated numerically:
\begin{align}
 \alpha_0 &= 1,424987 \mathrm{,}\; \alpha_1 = 0,992249 \mathrm{,} \\ 
 \alpha_2 &= 1,000198 \mathrm{,}\; \alpha_3 = 0,999994 \mathrm{,} \\
 \alpha_n &\approx 1 \text{ for all other } n.
\end{align}


Applying a force on the damped harmonic oscillator, that attacks only at the end of the reed ($x = l$), we can formulate the following differential equation:
\begin{equation}
 \frac{\partial^2z}{\partial t^2} + \frac{d^2}{12} v_Y \frac{\partial^4z}{\partial x^4} = f_0 e^{i\omega t}\delta(x - l)
\end{equation}

With only a small damping factor we can approximate the vibration modes around the resonance frequencies with a Lorentzian.
The relative amplitude can then be approximately described by 

\begin{align}
 \left| \frac{\zeta(\omega)}{\zeta(\omega_{R,n})} \right| &\approx \frac{\gamma \omega_{0,n)}}{\sqrt{(\omega_{0,n}^2 - \omega^2)^2 + \gamma^2\omega_{0,n}^2}} \\
 n &= 0, 1, 2,\dots \; \text{.}
\end{align}






\chapter{Setup}
\label{sec:set}
% Schöne Graphiken hier
% Ausführlich weil einfach!

\widegraph[./plot/setup]{Schematic Setup of the vibrating reed experiment. The components are: 1: Reed, 2: electrodes, 3: brass shielding, 4: Pt1000 thermometer, 5: Peltier element, 6: heat exchange, 7: fan, 8: computer, 9: \LIA, 10: frequency-generator, 11: battery. }{fig:set}

The core element of our experiment is the reed (1 in figure \ref{fig:set}).  It is a small metal plate that is fixed to the ground between two copper blocks.
The free end is placed between two electrodes (2) as can be seen in figure \ref{fig:set}. Its dimensions are approximately 
\SI{2}{\centi\meter} in length, \SI{5}{\milli\meter} in width and \SI{200}{\micro\meter} in depth. A brass shielding (3) prevents outside effects to influence the measurement.
Below the reed is a  (4), a Peltier-Element for temperature control (5), a heat exchange unit (6) and a fan (7). 
The experiment is controlled via a computer (8), which can manipulate parameters for the used \LIA (9) and frequency-generator (10). A constant voltage for the experiment is supplied by a battery (11).
For more information on the \LIA see section \ref{sec:lia}

The Reed is forced to vibrate by positioning an electrode next to it at a distance of $g_d$, and then applying a Voltage $U=U_0\operatorname{cos}(\omega t)$ to it. The reed and the electrode form a condensator with area of $S$. Thus a force of $F = \frac{1}{2} \frac{C_aU^2}{g_a}$ is acting on the reed, with $C_a = \epsilon_0 \frac{S}{g_a}$. As $U$ is time dependent we finally find 
\begin{equation}
 F(t) = \frac{1}{4} C_a \frac{U_0^2}{g_a}(1 + \operatorname{cos}(2\omega t)) \; \text{.}
\end{equation}

It is worth noting that the reed vibrates with twice the frequency than the applied voltage.

To measure the frequency of the reed a second electrode is put on the other side of the reed, which also forms a condensator with the capacity $C = \frac{C_d}{1-\frac{z}{g_d}} \approx C_d(1+\frac{z}{g_d}$, with $C_d = \epsilon_0 \frac{S}{g_g}$.

In the end we measure the voltage $U_b$, as shown in figure \ref{fig:el}. As described in the experiment script \cite{fp75} we can approximate the voltage with 

\begin{equation}
 U_d \approx U_b\frac{z}{g_d}\frac{C_d}{C_L} \; \text{.}
\end{equation}

All material and setup parameters for this experiment can be found in table \ref{tab:par}.
\begin{table*}[ht!]
\centering
 \begin{tabular}{c|c|l}
 Parameter	& Values					& Description \\
 \hline\hline
  $l$		& \SI{2}{\centi\meter}				& reed length \\
  $b$		& \SI{5}{\milli\meter}				& reed width \\
  $d$		& \SI{200}{\micro\meter}			& reed thickness \\
  $g_a$		& \SI{100}{\micro\meter}			& distance forcing electrode to reed \\
  $g_d$		& \SI{100}{\micro\meter}			& distance detection electrode to reed \\
  $D_a = D_d$	& \SI{3}{\milli\meter}				& diameter of electrodes \\
  $U_a = U_0$	& \SI{10}{\volt}				& amplitude of the forcing voltage \\
  $U_B$		& \SI{108}{\volt}				& battery voltage \\
  $E$		& \SI{98}{\giga\newton\per\meter\squared}	& elasticity of brass \\
  $\rho$	& \SI{8.5e3}{\kilo\gram\per\cubic\meter}	& density of brass \\
  $C_L$		& \SI{300}{\pico\farad}				& parasitical line capacity \\
  $R$		& \SI{300}{\mega\ohm}				& additional resistor \\
  \hline\hline
 \end{tabular}
 \caption{Material and setup parameters for the experiment.}
 \label{tab:par}
\end{table*}

\widegraph[./plot/electric]{Electrical configuration of the vibrating reed experiment}{fig:el}

\section{The \LIA}
\label{sec:lia}

As will be shown in section \ref{sec:par} the signal we want to detect is small compared to the noise that accompanies it. Thus we need a \LIA to detect it:

A \LIA relies on the PED, the ``phase sensitive detector''. Not only the Signal, but also a reference is supplied for the PED, containing information about frequency and phase of the signal. With the ``Comparer'' a rectangular signal is generated, which in turn operates on a switch in a periodic way. If the switch is in ``+1'' mode the original signal, if in ``-1'' mode, its invert  will be found at the exit. Using a low pass filter the first fourier Term of the resulting signal is selected, which is constant. Thus in the end, we get a voltage proportional to the input amplitude: $U \propto A\operatorname{cos}(\phi)$.
To get the maximum amplitude at the exit we must set the relative phase $\phi$ such that $\operatorname{cos}(\phi) = 1$. As this is tedious we use a \DLIA. It generates to rectangular signals out of the reference signal with one being moved by $\pi/2$. Also the input signal is doubled and mixed with these two rectangular signals. Thus at the exit we have two signals of constant voltage:

\begin{align}
 U_1 &= A\operatorname{cos}(\phi) \; \text{and} \\
 U_2 &= A\operatorname{sin}(\phi) \; \text{.}
\end{align}

From this we calculate the measured amplitude $A$ and the phase $\phi$:

\begin{align}
 A &= \sqrt{U_1^2 + U_2^2} \; \text{,} \\
 \phi &= \operatorname{arctan}\left( \frac{U_2}{U_1} \right) \; \text{.}
\end{align}

\section{Electronic Setup}
\label{sec:el}

A schematic of the electronic setup can be found in figure \ref{fig:el}. 

The frequency generator excites the reed with a frequency of $\nu / 2$ to oscillations with frequency $\nu$. As described before the signal is analyzed with a \DLIA and the necessary constant voltage is supplied by a battery. The excitation signal (used for the forcing) is supplied at the reference input of the \LIA. 

Frequency generator and \LIA can be controlled and the output signals read from a computer via a GPIB interface (General Purpose Interface Bus).

Heating and cooling of the experiment is done via a Peltier element, the current temperature can be read from a Pt1000 thermometer. Both are controlled via the AD-Ports of the \LIAn.


\widegraph[./plot/LIA]{Schematic Setup of a \LIAn.}{fig:lia}

\section{Parameter estimation}
\label{sec:par}
% Abschätzungen der verschiedenen Parameter

Prior to the experiment we need to estimate certain parameters to the reed, we use, be abled to run the experiment. These are the maximum Voltage $U_{d,max}$, which shows, 
how small our expected signals are and the base Eigenfrequency $\nu_0$, where we have to look for the Lorentz-peaks, we want to observe.


% U_d,maximum
The maximum
% nu_0



\chapter{Methods}
\label{sec:meth}
% Hier kommen die genutzten VIs rein und die Fitmethoden




\chapter{Results}
\label{sec:res}
% Kurz und knackig
% Alle aussagekräftigen graphen

\chapter{Discussion}
\label{sec:disk}
% Wieder ausführlich
% Siehe Protokoll

%\end{multicols}

\onecolumn
\bibliographystyle{plainnat}
\bibliography{CuD_long}
\end{document}

